\subsubsection{Caso de Uso: Crear Usuario}
\begin{table}[H]
    \renewcommand{\arraystretch}{1.3}
    \begin{tabularx}{\linewidth}{|>{\centering\arraybackslash}m{2.5cm}|>{\centering\arraybackslash}m{6.9cm}|>{\centering\arraybackslash}m{1.94cm}|}
        \hline
        \rowcolor{\headerColor}\textbf{Caso de Uso} & \textbf{Crear Usuario} & \textbf{CU-01} \\
        \hline
        \textbf{Actores} & \multicolumn{2}{|>{\raggedright\arraybackslash}X|}{Administrador}\\
        \hline
        \textbf{Tipo} & \multicolumn{2}{|>{\raggedright\arraybackslash}X|}{Primario} \\
        \hline
   \end{tabularx}
   \vspace{-1.1em}
  \begin{tabularx}{\linewidth}{|>{\centering\arraybackslash}m{2.5cm}|>{\centering\arraybackslash}m{4.42cm}|>{\centering\arraybackslash}m{4.42cm}|}
      \textbf{Referencias} & RF-2 & \\
      \hline
      \textbf{Precondición} & \multicolumn{2}{|>{\raggedright\arraybackslash}X|}{El nombre de usuario no debe pertenecer a un usuario existente} \\
      \hline
      \textbf{Postcondición} & \multicolumn{2}{|>{\raggedright\arraybackslash}X|}{El usuario se habrá creado correctamente} \\
      \hline
    \end{tabularx}
\end{table}
\vspace{-1em}
\begin{table}[H]
    \begin{tabularx}{\linewidth}{|>{\centering\arraybackslash}m{12.2cm}|}
      \hline
      \rowcolor{\headerColor}\textbf{Propósito} \\
      \hline
      Crear un nuevo usuario \\
      \hline
    \end{tabularx}
\end{table}
\vspace{-1em}
\begin{table}[H]
    \begin{tabularx}{\linewidth}{|>{\centering\arraybackslash}m{12.2cm}|}
      \hline
      \rowcolor{\headerColor}\textbf{Resumen} \\
      \hline
      Un administrador creará un nuevo usuario desde el panel de administración \\
      \hline
    \end{tabularx}
\end{table}
\vspace{-1em}
\begin{tabularx}{\linewidth}{
    |>{\centering\arraybackslash}p{0.3cm}
    |>{\raggedright\arraybackslash}p{5.1cm}
    |>{\centering\arraybackslash}p{0.3cm}
    |>{\raggedright\arraybackslash}p{5.1cm}|
  }
    \hline
    \multicolumn{4}{|>{\centering\arraybackslash}m{12.2cm}|}{\cellcolor{\headerColor}\textbf{Curso Normal}} \\
    \hline
    \endfirsthead
      1 & Administrador: Accede al panel de administración &  &  \\
      \hline
      2 & Administrador: Selecciona la creación de un nuevo y rellena los datos que se le solicitan &  &  \\
      \hline
       &  & 3 & Se comprueba que el nombre de usuario sea válido \\
      \hline
       &  & 4 & Se crea un nuevo usuario en la base de datos \\
      \hline
      5 & Administrador: Recibe una respuesta existosa &  &  \\
      \hline
    \multicolumn{4}{|>{\centering\arraybackslash}m{12.2cm}|}{\cellcolor{\headerColor}\textbf{Curso Alterno}} \\
    \hline
      3a & \multicolumn{3}{|>{\raggedright\arraybackslash}X|}{Si ya existe un usuario con el nombre de usuario especificado se devolverá un error} \\
      \hline
\end{tabularx}
\vspace{-1em}
\begin{table}[H]
    \begin{tabularx}{\linewidth}{
      |>{\centering\arraybackslash}p{2.4cm}
      |>{\raggedright\arraybackslash}p{3cm}
      |>{\centering\arraybackslash}p{2.4cm}
      |>{\raggedright\arraybackslash}p{3cm}|
    }
        \hline
        \multicolumn{4}{|>{\centering\arraybackslash}m{12.2cm}|}{\cellcolor{\headerColor}\textbf{Otros Datos}} \\
        \hline
        \textbf{Frecuencia esperada} & Muy Baja & \textbf{Rendimiento} & Alta \\
        \hline
        \textbf{Importancia} & Alta & \textbf{Urgencia} & Alta \\
        \hline
        \textbf{Estado} & Realizado & \textbf{Estabilidad} & Alta \\
        \hline
        \multicolumn{4}{|>{\centering\arraybackslash}m{12.2cm}|}{\cellcolor{\headerColor}\textbf{Comentarios}} \\
        \hline
        \multicolumn{4}{|>{\centering\arraybackslash}X|}{}\\
        \hline
    \end{tabularx}
\end{table}
\subsubsection{Caso de Uso: Eliminar Usuario}
\begin{table}[H]
    \renewcommand{\arraystretch}{1.3}
    \begin{tabularx}{\linewidth}{|>{\centering\arraybackslash}m{2.5cm}|>{\centering\arraybackslash}m{6.9cm}|>{\centering\arraybackslash}m{1.94cm}|}
        \hline
        \rowcolor{\headerColor}\textbf{Caso de Uso} & \textbf{Eliminar Usuario} & \textbf{CU-02} \\
        \hline
        \textbf{Actores} & \multicolumn{2}{|>{\raggedright\arraybackslash}X|}{Administrador}\\
        \hline
        \textbf{Tipo} & \multicolumn{2}{|>{\raggedright\arraybackslash}X|}{Primario} \\
        \hline
   \end{tabularx}
   \vspace{-1.1em}
  \begin{tabularx}{\linewidth}{|>{\centering\arraybackslash}m{2.5cm}|>{\centering\arraybackslash}m{4.42cm}|>{\centering\arraybackslash}m{4.42cm}|}
      \textbf{Referencias} & RF-3 & CU-7, CU-11, CU-14\\
      \hline
      \textbf{Precondición} & \multicolumn{2}{|>{\raggedright\arraybackslash}X|}{El usuario existe en el sistema} \\
      \hline
      \textbf{Postcondición} & \multicolumn{2}{|>{\raggedright\arraybackslash}X|}{El usuario se habrá borrado del sistema} \\
      \hline
    \end{tabularx}
\end{table}
\vspace{-1em}
\begin{table}[H]
    \begin{tabularx}{\linewidth}{|>{\centering\arraybackslash}m{12.2cm}|}
      \hline
      \rowcolor{\headerColor}\textbf{Propósito} \\
      \hline
      Borrar un usuario existente \\
      \hline
    \end{tabularx}
\end{table}
\vspace{-1em}
\begin{table}[H]
    \begin{tabularx}{\linewidth}{|>{\centering\arraybackslash}m{12.2cm}|}
      \hline
      \rowcolor{\headerColor}\textbf{Resumen} \\
      \hline
      El administrador borrará un usuario existente desde el panel de administración \\
      \hline
    \end{tabularx}
\end{table}
\vspace{-1em}
\begin{tabularx}{\linewidth}{
    |>{\centering\arraybackslash}p{0.3cm}
    |>{\raggedright\arraybackslash}p{5.1cm}
    |>{\centering\arraybackslash}p{0.3cm}
    |>{\raggedright\arraybackslash}p{5.1cm}|
  }
    \hline
    \multicolumn{4}{|>{\centering\arraybackslash}m{12.2cm}|}{\cellcolor{\headerColor}\textbf{Curso Normal}} \\
    \hline
    \endfirsthead
      1 & Administrador: Accede al panel de administración &  &  \\
      \hline
      2 & Administrador: Selecciona la la eliminación de un usuario &  &  \\
      \hline
       &  & 3 & Se borran los recursos asociados al usuario (datasets, modelos y clasificaciones) \\
      \hline
       &  & 4 & Se elimina al usuario de la base de datos \\
      \hline
      5 & Administrador: Recibe una respuesta existosa &  &  \\
      \hline
    \multicolumn{4}{|>{\centering\arraybackslash}m{12.2cm}|}{\cellcolor{\headerColor}\textbf{Curso Alterno}} \\
    \hline
       & \multicolumn{3}{|>{\raggedright\arraybackslash}X|}{} \\
      \hline
\end{tabularx}
\vspace{-1em}
\begin{table}[H]
    \begin{tabularx}{\linewidth}{
      |>{\centering\arraybackslash}p{2.4cm}
      |>{\raggedright\arraybackslash}p{3cm}
      |>{\centering\arraybackslash}p{2.4cm}
      |>{\raggedright\arraybackslash}p{3cm}|
    }
        \hline
        \multicolumn{4}{|>{\centering\arraybackslash}m{12.2cm}|}{\cellcolor{\headerColor}\textbf{Otros Datos}} \\
        \hline
        \textbf{Frecuencia esperada} & Muy baja & \textbf{Rendimiento} & Alta \\
        \hline
        \textbf{Importancia} & Media & \textbf{Urgencia} & Baja \\
        \hline
        \textbf{Estado} & Realizado & \textbf{Estabilidad} & Alta \\
        \hline
        \multicolumn{4}{|>{\centering\arraybackslash}m{12.2cm}|}{\cellcolor{\headerColor}\textbf{Comentarios}} \\
        \hline
        \multicolumn{4}{|>{\centering\arraybackslash}X|}{}\\
        \hline
    \end{tabularx}
\end{table}
\subsubsection{Caso de Uso: Iniciar sesión}
\begin{table}[H]
    \renewcommand{\arraystretch}{1.3}
    \begin{tabularx}{\linewidth}{|>{\centering\arraybackslash}m{2.5cm}|>{\centering\arraybackslash}m{6.9cm}|>{\centering\arraybackslash}m{1.94cm}|}
        \hline
        \rowcolor{\headerColor}\textbf{Caso de Uso} & \textbf{Iniciar sesión} & \textbf{CU-03} \\
        \hline
        \textbf{Actores} & \multicolumn{2}{|>{\raggedright\arraybackslash}X|}{Visitante, Usuario(Secundario)}\\
        \hline
        \textbf{Tipo} & \multicolumn{2}{|>{\raggedright\arraybackslash}X|}{Primario} \\
        \hline
   \end{tabularx}
   \vspace{-1.1em}
  \begin{tabularx}{\linewidth}{|>{\centering\arraybackslash}m{2.5cm}|>{\centering\arraybackslash}m{4.42cm}|>{\centering\arraybackslash}m{4.42cm}|}
      \textbf{Referencias} & RF-4, RF-5 & \\
      \hline
      \textbf{Precondición} & \multicolumn{2}{|>{\raggedright\arraybackslash}X|}{El usuario no está autenticado en el sistema} \\
      \hline
      \textbf{Postcondición} & \multicolumn{2}{|>{\raggedright\arraybackslash}X|}{El rol del actor pasará a ser Usuario} \\
      \hline
    \end{tabularx}
\end{table}
\vspace{-1em}
\begin{table}[H]
    \begin{tabularx}{\linewidth}{|>{\centering\arraybackslash}m{12.2cm}|}
      \hline
      \rowcolor{\headerColor}\textbf{Propósito} \\
      \hline
      Iniciar sesión en la aplicación \\
      \hline
    \end{tabularx}
\end{table}
\vspace{-1em}
\begin{table}[H]
    \begin{tabularx}{\linewidth}{|>{\centering\arraybackslash}m{12.2cm}|}
      \hline
      \rowcolor{\headerColor}\textbf{Resumen} \\
      \hline
      Un visitante iniciará sesión en el sistema con una cuenta existente \\
      \hline
    \end{tabularx}
\end{table}
\vspace{-1em}
\begin{tabularx}{\linewidth}{
    |>{\centering\arraybackslash}p{0.3cm}
    |>{\raggedright\arraybackslash}p{5.1cm}
    |>{\centering\arraybackslash}p{0.3cm}
    |>{\raggedright\arraybackslash}p{5.1cm}|
  }
    \hline
    \multicolumn{4}{|>{\centering\arraybackslash}m{12.2cm}|}{\cellcolor{\headerColor}\textbf{Curso Normal}} \\
    \hline
    \endfirsthead
      1 & Visitante: Entrará a la página de inicio de sesión &  &  \\
      \hline
      2 & Visitante: Introducirá sus datos de usuario en el formulario &  &  \\
      \hline
       &  & 3 & Se valida que el nombre de usuario y contraseña sean correctos \\
      \hline
       &  & 4 & Se guarda el token de sesión en el cliente \\
      \hline
      5 & Usuario: Se redirigirá a la pantalla de inicio del usuario &  &  \\
      \hline
    \multicolumn{4}{|>{\centering\arraybackslash}m{12.2cm}|}{\cellcolor{\headerColor}\textbf{Curso Alterno}} \\
    \hline
      3a & \multicolumn{3}{|>{\raggedright\arraybackslash}X|}{Si el nombre de usuario o contraseña no son válidos se devolverá un error al usuario y se mantendrá en la página de inicio de sesión} \\
      \hline
\end{tabularx}
\vspace{-1em}
\begin{table}[H]
    \begin{tabularx}{\linewidth}{
      |>{\centering\arraybackslash}p{2.4cm}
      |>{\raggedright\arraybackslash}p{3cm}
      |>{\centering\arraybackslash}p{2.4cm}
      |>{\raggedright\arraybackslash}p{3cm}|
    }
        \hline
        \multicolumn{4}{|>{\centering\arraybackslash}m{12.2cm}|}{\cellcolor{\headerColor}\textbf{Otros Datos}} \\
        \hline
        \textbf{Frecuencia esperada} & Baja & \textbf{Rendimiento} & Alto \\
        \hline
        \textbf{Importancia} & Alta & \textbf{Urgencia} & Alta \\
        \hline
        \textbf{Estado} & Realizado & \textbf{Estabilidad} & Alta \\
        \hline
        \multicolumn{4}{|>{\centering\arraybackslash}m{12.2cm}|}{\cellcolor{\headerColor}\textbf{Comentarios}} \\
        \hline
        \multicolumn{4}{|>{\centering\arraybackslash}X|}{El token de autenticación se guardará en el cliente, permitiendo al usuario mantener su sesión iniciada sin tener que repetir el proceso cada vez}\\
        \hline
    \end{tabularx}
\end{table}
\subsubsection{Caso de Uso: Modificar Contraseña}
\begin{table}[H]
    \renewcommand{\arraystretch}{1.3}
    \begin{tabularx}{\linewidth}{|>{\centering\arraybackslash}m{2.5cm}|>{\centering\arraybackslash}m{6.9cm}|>{\centering\arraybackslash}m{1.94cm}|}
        \hline
        \rowcolor{\headerColor}\textbf{Caso de Uso} & \textbf{Modificar Contraseña} & \textbf{CU-04} \\
        \hline
        \textbf{Actores} & \multicolumn{2}{|>{\raggedright\arraybackslash}X|}{Usuario}\\
        \hline
        \textbf{Tipo} & \multicolumn{2}{|>{\raggedright\arraybackslash}X|}{Primario} \\
        \hline
   \end{tabularx}
   \vspace{-1.1em}
  \begin{tabularx}{\linewidth}{|>{\centering\arraybackslash}m{2.5cm}|>{\centering\arraybackslash}m{4.42cm}|>{\centering\arraybackslash}m{4.42cm}|}
      \textbf{Referencias} & RF-20 & \\
      \hline
      \textbf{Precondición} & \multicolumn{2}{|>{\raggedright\arraybackslash}X|}{} \\
      \hline
      \textbf{Postcondición} & \multicolumn{2}{|>{\raggedright\arraybackslash}X|}{La contraseña del usuario se habrá actualizado en la base de datos} \\
      \hline
    \end{tabularx}
\end{table}
\vspace{-1em}
\begin{table}[H]
    \begin{tabularx}{\linewidth}{|>{\centering\arraybackslash}m{12.2cm}|}
      \hline
      \rowcolor{\headerColor}\textbf{Propósito} \\
      \hline
      Modificar la contraseña del usuario \\
      \hline
    \end{tabularx}
\end{table}
\vspace{-1em}
\begin{table}[H]
    \begin{tabularx}{\linewidth}{|>{\centering\arraybackslash}m{12.2cm}|}
      \hline
      \rowcolor{\headerColor}\textbf{Resumen} \\
      \hline
      Un usuario podrá modificar su contraseña desde la página de usuario \\
      \hline
    \end{tabularx}
\end{table}
\vspace{-1em}
\begin{tabularx}{\linewidth}{
    |>{\centering\arraybackslash}p{0.3cm}
    |>{\raggedright\arraybackslash}p{5.1cm}
    |>{\centering\arraybackslash}p{0.3cm}
    |>{\raggedright\arraybackslash}p{5.1cm}|
  }
    \hline
    \multicolumn{4}{|>{\centering\arraybackslash}m{12.2cm}|}{\cellcolor{\headerColor}\textbf{Curso Normal}} \\
    \hline
    \endfirsthead
      1 & Usuario: Entrará al panel de gestión de usuario &  &  \\
      \hline
      2 & Usuario: Rellenará el formulario de cambio de contraseña especificando su contraseña actual y la contraseá nueva &  &  \\
      \hline
       &  & 3 & Se valida que la contraseña proporcionada sea correcta \\
      \hline
       &  & 4 & Se modifica la contraseña del usuario en la base de datos \\
      \hline
      5 & Usuario: Recibirá una respuesta exitosa del sistema &  &  \\
      \hline
    \multicolumn{4}{|>{\centering\arraybackslash}m{12.2cm}|}{\cellcolor{\headerColor}\textbf{Curso Alterno}} \\
    \hline
      3a & \multicolumn{3}{|>{\raggedright\arraybackslash}X|}{Si la contraseña actual proporcionada no es correcta se devolverá un mensaje de error al usuario} \\
      \hline
\end{tabularx}
\vspace{-1em}
\begin{table}[H]
    \begin{tabularx}{\linewidth}{
      |>{\centering\arraybackslash}p{2.4cm}
      |>{\raggedright\arraybackslash}p{3cm}
      |>{\centering\arraybackslash}p{2.4cm}
      |>{\raggedright\arraybackslash}p{3cm}|
    }
        \hline
        \multicolumn{4}{|>{\centering\arraybackslash}m{12.2cm}|}{\cellcolor{\headerColor}\textbf{Otros Datos}} \\
        \hline
        \textbf{Frecuencia esperada} & Muy baja & \textbf{Rendimiento} & Alta \\
        \hline
        \textbf{Importancia} & Alta & \textbf{Urgencia} & Baja \\
        \hline
        \textbf{Estado} & Realizado & \textbf{Estabilidad} & Alta \\
        \hline
        \multicolumn{4}{|>{\centering\arraybackslash}m{12.2cm}|}{\cellcolor{\headerColor}\textbf{Comentarios}} \\
        \hline
        \multicolumn{4}{|>{\centering\arraybackslash}X|}{}\\
        \hline
    \end{tabularx}
\end{table}
\subsubsection{Caso de Uso: Cerrar Sesión}
\begin{table}[H]
    \renewcommand{\arraystretch}{1.3}
    \begin{tabularx}{\linewidth}{|>{\centering\arraybackslash}m{2.5cm}|>{\centering\arraybackslash}m{6.9cm}|>{\centering\arraybackslash}m{1.94cm}|}
        \hline
        \rowcolor{\headerColor}\textbf{Caso de Uso} & \textbf{Cerrar Sesión} & \textbf{CU-05} \\
        \hline
        \textbf{Actores} & \multicolumn{2}{|>{\raggedright\arraybackslash}X|}{Usuario, Visitante(Secundario)}\\
        \hline
        \textbf{Tipo} & \multicolumn{2}{|>{\raggedright\arraybackslash}X|}{Primario} \\
        \hline
   \end{tabularx}
   \vspace{-1.1em}
  \begin{tabularx}{\linewidth}{|>{\centering\arraybackslash}m{2.5cm}|>{\centering\arraybackslash}m{4.42cm}|>{\centering\arraybackslash}m{4.42cm}|}
      \textbf{Referencias} & RF-6 & \\
      \hline
      \textbf{Precondición} & \multicolumn{2}{|>{\raggedright\arraybackslash}X|}{} \\
      \hline
      \textbf{Postcondición} & \multicolumn{2}{|>{\raggedright\arraybackslash}X|}{El rol del actor pasará a ser Visitante} \\
      \hline
    \end{tabularx}
\end{table}
\vspace{-1em}
\begin{table}[H]
    \begin{tabularx}{\linewidth}{|>{\centering\arraybackslash}m{12.2cm}|}
      \hline
      \rowcolor{\headerColor}\textbf{Propósito} \\
      \hline
      Permitir cerrar sesión al Usuario \\
      \hline
    \end{tabularx}
\end{table}
\vspace{-1em}
\begin{table}[H]
    \begin{tabularx}{\linewidth}{|>{\centering\arraybackslash}m{12.2cm}|}
      \hline
      \rowcolor{\headerColor}\textbf{Resumen} \\
      \hline
      El usuario cerrará sesión en la aplicación \\
      \hline
    \end{tabularx}
\end{table}
\vspace{-1em}
\begin{tabularx}{\linewidth}{
    |>{\centering\arraybackslash}p{0.3cm}
    |>{\raggedright\arraybackslash}p{5.1cm}
    |>{\centering\arraybackslash}p{0.3cm}
    |>{\raggedright\arraybackslash}p{5.1cm}|
  }
    \hline
    \multicolumn{4}{|>{\centering\arraybackslash}m{12.2cm}|}{\cellcolor{\headerColor}\textbf{Curso Normal}} \\
    \hline
    \endfirsthead
      1 & Usuario: Interactuará con el botón de cierre de sesión &  &  \\
      \hline
       &  & 2 & Se borrará del cliente el token de autenticación \\
      \hline
       &  & 3 & Se invalidará el token de autenticación en la base de datos \\
      \hline
      4 & Visitante: Se le redigirá a la página de inicio de sesión &  &  \\
      \hline
    \multicolumn{4}{|>{\centering\arraybackslash}m{12.2cm}|}{\cellcolor{\headerColor}\textbf{Curso Alterno}} \\
    \hline
       & \multicolumn{3}{|>{\raggedright\arraybackslash}X|}{} \\
      \hline
\end{tabularx}
\vspace{-1em}
\begin{table}[H]
    \begin{tabularx}{\linewidth}{
      |>{\centering\arraybackslash}p{2.4cm}
      |>{\raggedright\arraybackslash}p{3cm}
      |>{\centering\arraybackslash}p{2.4cm}
      |>{\raggedright\arraybackslash}p{3cm}|
    }
        \hline
        \multicolumn{4}{|>{\centering\arraybackslash}m{12.2cm}|}{\cellcolor{\headerColor}\textbf{Otros Datos}} \\
        \hline
        \textbf{Frecuencia esperada} & Baja & \textbf{Rendimiento} & Alto \\
        \hline
        \textbf{Importancia} & Alta & \textbf{Urgencia} & Baja \\
        \hline
        \textbf{Estado} & Realizado & \textbf{Estabilidad} & Alta \\
        \hline
        \multicolumn{4}{|>{\centering\arraybackslash}m{12.2cm}|}{\cellcolor{\headerColor}\textbf{Comentarios}} \\
        \hline
        \multicolumn{4}{|>{\centering\arraybackslash}X|}{El token de autenticación se borrará del cliente}\\
        \hline
    \end{tabularx}
\end{table}
\subsubsection{Caso de Uso: Subir Dataset}
\begin{table}[H]
    \renewcommand{\arraystretch}{1.3}
    \begin{tabularx}{\linewidth}{|>{\centering\arraybackslash}m{2.5cm}|>{\centering\arraybackslash}m{6.9cm}|>{\centering\arraybackslash}m{1.94cm}|}
        \hline
        \rowcolor{\headerColor}\textbf{Caso de Uso} & \textbf{Subir Dataset} & \textbf{CU-06} \\
        \hline
        \textbf{Actores} & \multicolumn{2}{|>{\raggedright\arraybackslash}X|}{Usuario}\\
        \hline
        \textbf{Tipo} & \multicolumn{2}{|>{\raggedright\arraybackslash}X|}{Primario} \\
        \hline
   \end{tabularx}
   \vspace{-1.1em}
  \begin{tabularx}{\linewidth}{|>{\centering\arraybackslash}m{2.5cm}|>{\centering\arraybackslash}m{4.42cm}|>{\centering\arraybackslash}m{4.42cm}|}
      \textbf{Referencias} & RF-7, RF-8, RNF-2 & \\
      \hline
      \textbf{Precondición} & \multicolumn{2}{|>{\raggedright\arraybackslash}X|}{El usuario no tiene subido ese dataset en el sistema} \\
      \hline
      \textbf{Postcondición} & \multicolumn{2}{|>{\raggedright\arraybackslash}X|}{Se subirá el dataset al sistema} \\
      \hline
    \end{tabularx}
\end{table}
\vspace{-1em}
\begin{table}[H]
    \begin{tabularx}{\linewidth}{|>{\centering\arraybackslash}m{12.2cm}|}
      \hline
      \rowcolor{\headerColor}\textbf{Propósito} \\
      \hline
      El usuario podrá subir un dataset al sistema \\
      \hline
    \end{tabularx}
\end{table}
\vspace{-1em}
\begin{table}[H]
    \begin{tabularx}{\linewidth}{|>{\centering\arraybackslash}m{12.2cm}|}
      \hline
      \rowcolor{\headerColor}\textbf{Resumen} \\
      \hline
      El usuario seleccionará subir un nuevo dataset al sistema, ya sea desde un fichero local o cargando el dataset desde HuggingFace \\
      \hline
    \end{tabularx}
\end{table}
\vspace{-1em}
\begin{tabularx}{\linewidth}{
    |>{\centering\arraybackslash}p{0.3cm}
    |>{\raggedright\arraybackslash}p{5.1cm}
    |>{\centering\arraybackslash}p{0.3cm}
    |>{\raggedright\arraybackslash}p{5.1cm}|
  }
    \hline
    \multicolumn{4}{|>{\centering\arraybackslash}m{12.2cm}|}{\cellcolor{\headerColor}\textbf{Curso Normal}} \\
    \hline
    \endfirsthead
      1 & Usuario: Seleccionará añadir un nuevo dataset &  &  \\
      \hline
      2 & Usuario: Seleccionará el dataset que quiera subir &  &  \\
      \hline
       &  & 3 & Se creará una nueva entrada en la base de datos con el dataset creado \\
      \hline
      4 & Usuario: Recibirá una respuesta exitosa &  &  \\
      \hline
    \multicolumn{4}{|>{\centering\arraybackslash}m{12.2cm}|}{\cellcolor{\headerColor}\textbf{Curso Alterno}} \\
    \hline
      2a & \multicolumn{3}{|>{\raggedright\arraybackslash}X|}{El usuario podrá subir el dataset desde un fichero local, pudiendo marcar este como privado} \\
      \hline
      2b & \multicolumn{3}{|>{\raggedright\arraybackslash}X|}{El usuario podrá buscar un dataset existente en HuggingFace y subirlo al sistema} \\
      \hline
      3a & \multicolumn{3}{|>{\raggedright\arraybackslash}X|}{Si el dataset se ha cargado desde un fichero local se guardará en el servidor} \\
      \hline
\end{tabularx}
\vspace{-1em}
\begin{table}[H]
    \begin{tabularx}{\linewidth}{
      |>{\centering\arraybackslash}p{2.4cm}
      |>{\raggedright\arraybackslash}p{3cm}
      |>{\centering\arraybackslash}p{2.4cm}
      |>{\raggedright\arraybackslash}p{3cm}|
    }
        \hline
        \multicolumn{4}{|>{\centering\arraybackslash}m{12.2cm}|}{\cellcolor{\headerColor}\textbf{Otros Datos}} \\
        \hline
        \textbf{Frecuencia esperada} & Media & \textbf{Rendimiento} & Medio \\
        \hline
        \textbf{Importancia} & Alta & \textbf{Urgencia} & Alta \\
        \hline
        \textbf{Estado} & Realizado & \textbf{Estabilidad} & Alta \\
        \hline
        \multicolumn{4}{|>{\centering\arraybackslash}m{12.2cm}|}{\cellcolor{\headerColor}\textbf{Comentarios}} \\
        \hline
        \multicolumn{4}{|>{\centering\arraybackslash}X|}{}\\
        \hline
    \end{tabularx}
\end{table}
\subsubsection{Caso de Uso: Borrar Dataset}
\begin{table}[H]
    \renewcommand{\arraystretch}{1.3}
    \begin{tabularx}{\linewidth}{|>{\centering\arraybackslash}m{2.5cm}|>{\centering\arraybackslash}m{6.9cm}|>{\centering\arraybackslash}m{1.94cm}|}
        \hline
        \rowcolor{\headerColor}\textbf{Caso de Uso} & \textbf{Borrar Dataset} & \textbf{CU-07} \\
        \hline
        \textbf{Actores} & \multicolumn{2}{|>{\raggedright\arraybackslash}X|}{Usuario Propietario, Administrador}\\
        \hline
        \textbf{Tipo} & \multicolumn{2}{|>{\raggedright\arraybackslash}X|}{Primario} \\
        \hline
   \end{tabularx}
   \vspace{-1.1em}
  \begin{tabularx}{\linewidth}{|>{\centering\arraybackslash}m{2.5cm}|>{\centering\arraybackslash}m{4.42cm}|>{\centering\arraybackslash}m{4.42cm}|}
      \textbf{Referencias} & RF-15 & CU-02, CU-14\\
      \hline
      \textbf{Precondición} & \multicolumn{2}{|>{\raggedright\arraybackslash}X|}{El dataset tiene que existir en el sistema} \\
      \hline
      \textbf{Postcondición} & \multicolumn{2}{|>{\raggedright\arraybackslash}X|}{El dataset se borrará del sistema} \\
      \hline
    \end{tabularx}
\end{table}
\vspace{-1em}
\begin{table}[H]
    \begin{tabularx}{\linewidth}{|>{\centering\arraybackslash}m{12.2cm}|}
      \hline
      \rowcolor{\headerColor}\textbf{Propósito} \\
      \hline
      Borrar un dataset de un usuario \\
      \hline
    \end{tabularx}
\end{table}
\vspace{-1em}
\begin{table}[H]
    \begin{tabularx}{\linewidth}{|>{\centering\arraybackslash}m{12.2cm}|}
      \hline
      \rowcolor{\headerColor}\textbf{Resumen} \\
      \hline
      Un usuario o administrador podrá borrar un dataset existente en el sistema \\
      \hline
    \end{tabularx}
\end{table}
\vspace{-1em}
\begin{tabularx}{\linewidth}{
    |>{\centering\arraybackslash}p{0.3cm}
    |>{\raggedright\arraybackslash}p{5.1cm}
    |>{\centering\arraybackslash}p{0.3cm}
    |>{\raggedright\arraybackslash}p{5.1cm}|
  }
    \hline
    \multicolumn{4}{|>{\centering\arraybackslash}m{12.2cm}|}{\cellcolor{\headerColor}\textbf{Curso Normal}} \\
    \hline
    \endfirsthead
      1 & Usuario: Seleccionará el dataset a borrar y se le solicitará confirmación antes de borrarlo &  &  \\
      \hline
       &  & 2 & Se borrarán las clasificaciones asociadas al dataset \\
      \hline
       &  & 3 & Se borrarán los ficheros físicos asociados al dataset \\
      \hline
       &  & 4 & Se borrará el dataset de la base de datos \\
      \hline
      5 & Usuario: Recibirá una respuesta exitosa &  &  \\
      \hline
    \multicolumn{4}{|>{\centering\arraybackslash}m{12.2cm}|}{\cellcolor{\headerColor}\textbf{Curso Alterno}} \\
    \hline
       & \multicolumn{3}{|>{\raggedright\arraybackslash}X|}{} \\
      \hline
\end{tabularx}
\vspace{-1em}
\begin{table}[H]
    \begin{tabularx}{\linewidth}{
      |>{\centering\arraybackslash}p{2.4cm}
      |>{\raggedright\arraybackslash}p{3cm}
      |>{\centering\arraybackslash}p{2.4cm}
      |>{\raggedright\arraybackslash}p{3cm}|
    }
        \hline
        \multicolumn{4}{|>{\centering\arraybackslash}m{12.2cm}|}{\cellcolor{\headerColor}\textbf{Otros Datos}} \\
        \hline
        \textbf{Frecuencia esperada} & Baja & \textbf{Rendimiento} & Alto \\
        \hline
        \textbf{Importancia} & Alta & \textbf{Urgencia} & Media \\
        \hline
        \textbf{Estado} & Realizado & \textbf{Estabilidad} & Alta \\
        \hline
        \multicolumn{4}{|>{\centering\arraybackslash}m{12.2cm}|}{\cellcolor{\headerColor}\textbf{Comentarios}} \\
        \hline
        \multicolumn{4}{|>{\centering\arraybackslash}X|}{}\\
        \hline
    \end{tabularx}
\end{table}
\subsubsection{Caso de Uso: Listar datasets de un usuario}
\begin{table}[H]
    \renewcommand{\arraystretch}{1.3}
    \begin{tabularx}{\linewidth}{|>{\centering\arraybackslash}m{2.5cm}|>{\centering\arraybackslash}m{6.9cm}|>{\centering\arraybackslash}m{1.94cm}|}
        \hline
        \rowcolor{\headerColor}\textbf{Caso de Uso} & \textbf{Listar datasets de un usuario} & \textbf{CU-08} \\
        \hline
        \textbf{Actores} & \multicolumn{2}{|>{\raggedright\arraybackslash}X|}{Usuario}\\
        \hline
        \textbf{Tipo} & \multicolumn{2}{|>{\raggedright\arraybackslash}X|}{Primario} \\
        \hline
   \end{tabularx}
   \vspace{-1.1em}
  \begin{tabularx}{\linewidth}{|>{\centering\arraybackslash}m{2.5cm}|>{\centering\arraybackslash}m{4.42cm}|>{\centering\arraybackslash}m{4.42cm}|}
      \textbf{Referencias} & RF-11, RF-12 & \\
      \hline
      \textbf{Precondición} & \multicolumn{2}{|>{\raggedright\arraybackslash}X|}{El usuario del que se quieren listar los datasets debe existir} \\
      \hline
      \textbf{Postcondición} & \multicolumn{2}{|>{\raggedright\arraybackslash}X|}{} \\
      \hline
    \end{tabularx}
\end{table}
\vspace{-1em}
\begin{table}[H]
    \begin{tabularx}{\linewidth}{|>{\centering\arraybackslash}m{12.2cm}|}
      \hline
      \rowcolor{\headerColor}\textbf{Propósito} \\
      \hline
      Mostrar el listado de datasets que ha subido un usuario \\
      \hline
    \end{tabularx}
\end{table}
\vspace{-1em}
\begin{table}[H]
    \begin{tabularx}{\linewidth}{|>{\centering\arraybackslash}m{12.2cm}|}
      \hline
      \rowcolor{\headerColor}\textbf{Resumen} \\
      \hline
      Un usuario podrá ver el listado de datasets de cualquier usuario, siempre que sean públicos o sea propietario. \\
      \hline
    \end{tabularx}
\end{table}
\vspace{-1em}
\begin{tabularx}{\linewidth}{
    |>{\centering\arraybackslash}p{0.3cm}
    |>{\raggedright\arraybackslash}p{5.1cm}
    |>{\centering\arraybackslash}p{0.3cm}
    |>{\raggedright\arraybackslash}p{5.1cm}|
  }
    \hline
    \multicolumn{4}{|>{\centering\arraybackslash}m{12.2cm}|}{\cellcolor{\headerColor}\textbf{Curso Normal}} \\
    \hline
    \endfirsthead
      1 & Usuario: Seleccionará ver los datasets de un usuario en concreto &  &  \\
      \hline
       &  & 2 & Se comprobará que el usuario que ha solicitado los datasets es administrador o propietario \\
      \hline
       &  & 3 & Se obtendrán de la base de datos la lista de datasets relacionados con el usuario objetivo a los que el usuario solicitante tiene permiso \\
      \hline
      4 & Usuario: Recibirá una lista de datasets que ha subido el usuario especificado &  &  \\
      \hline
    \multicolumn{4}{|>{\centering\arraybackslash}m{12.2cm}|}{\cellcolor{\headerColor}\textbf{Curso Alterno}} \\
    \hline
      4a & \multicolumn{3}{|>{\raggedright\arraybackslash}X|}{Si el usuario es administrador o es el propietario de los datasets, la lista de datasets incluirá una opción para borrarlos del sistema} \\
      \hline
\end{tabularx}
\vspace{-1em}
\begin{table}[H]
    \begin{tabularx}{\linewidth}{
      |>{\centering\arraybackslash}p{2.4cm}
      |>{\raggedright\arraybackslash}p{3cm}
      |>{\centering\arraybackslash}p{2.4cm}
      |>{\raggedright\arraybackslash}p{3cm}|
    }
        \hline
        \multicolumn{4}{|>{\centering\arraybackslash}m{12.2cm}|}{\cellcolor{\headerColor}\textbf{Otros Datos}} \\
        \hline
        \textbf{Frecuencia esperada} & Alta & \textbf{Rendimiento} & Alto \\
        \hline
        \textbf{Importancia} & Alta & \textbf{Urgencia} & Alta \\
        \hline
        \textbf{Estado} & Realizado & \textbf{Estabilidad} & Alta \\
        \hline
        \multicolumn{4}{|>{\centering\arraybackslash}m{12.2cm}|}{\cellcolor{\headerColor}\textbf{Comentarios}} \\
        \hline
        \multicolumn{4}{|>{\centering\arraybackslash}X|}{}\\
        \hline
    \end{tabularx}
\end{table}
\subsubsection{Caso de Uso: Inspeccionar Dataset}
\begin{table}[H]
    \renewcommand{\arraystretch}{1.3}
    \begin{tabularx}{\linewidth}{|>{\centering\arraybackslash}m{2.5cm}|>{\centering\arraybackslash}m{6.9cm}|>{\centering\arraybackslash}m{1.94cm}|}
        \hline
        \rowcolor{\headerColor}\textbf{Caso de Uso} & \textbf{Inspeccionar Dataset} & \textbf{CU-09} \\
        \hline
        \textbf{Actores} & \multicolumn{2}{|>{\raggedright\arraybackslash}X|}{Usuario}\\
        \hline
        \textbf{Tipo} & \multicolumn{2}{|>{\raggedright\arraybackslash}X|}{Primario} \\
        \hline
   \end{tabularx}
   \vspace{-1.1em}
  \begin{tabularx}{\linewidth}{|>{\centering\arraybackslash}m{2.5cm}|>{\centering\arraybackslash}m{4.42cm}|>{\centering\arraybackslash}m{4.42cm}|}
      \textbf{Referencias} & RF-11, RF-12 & CU-13\\
      \hline
      \textbf{Precondición} & \multicolumn{2}{|>{\raggedright\arraybackslash}X|}{El dataset debe existir en el sistema} \\
      \hline
      \textbf{Postcondición} & \multicolumn{2}{|>{\raggedright\arraybackslash}X|}{} \\
      \hline
    \end{tabularx}
\end{table}
\vspace{-1em}
\begin{table}[H]
    \begin{tabularx}{\linewidth}{|>{\centering\arraybackslash}m{12.2cm}|}
      \hline
      \rowcolor{\headerColor}\textbf{Propósito} \\
      \hline
      Mostrar la información de un dataset \\
      \hline
    \end{tabularx}
\end{table}
\vspace{-1em}
\begin{table}[H]
    \begin{tabularx}{\linewidth}{|>{\centering\arraybackslash}m{12.2cm}|}
      \hline
      \rowcolor{\headerColor}\textbf{Resumen} \\
      \hline
      Un usuario podrá ver la información relativa a un dataset, una pequeña previsualización de los datos y la lista de clasificaciones realizadas a ese dataset \\
      \hline
    \end{tabularx}
\end{table}
\vspace{-1em}
\begin{tabularx}{\linewidth}{
    |>{\centering\arraybackslash}p{0.3cm}
    |>{\raggedright\arraybackslash}p{5.1cm}
    |>{\centering\arraybackslash}p{0.3cm}
    |>{\raggedright\arraybackslash}p{5.1cm}|
  }
    \hline
    \multicolumn{4}{|>{\centering\arraybackslash}m{12.2cm}|}{\cellcolor{\headerColor}\textbf{Curso Normal}} \\
    \hline
    \endfirsthead
      1 & Usuario: Accederá a la visualización de un dataset &  &  \\
      \hline
       &  & 2 & Se comprobará si el usuario tiene permisos para acceder al dataset \\
      \hline
       &  & 3 & Se obtendrá la información relacionada al dataset de la base de datos \\
      \hline
       &  & 4 & Se obtendrá una previsualización de los datos del dataset \\
      \hline
       &  & 5 & Se obtendrá la lista de clasificaciones realizadas al dataset \\
      \hline
      6 & Usuario: Recibirá la información relacionada con el dataset &  &  \\
      \hline
    \multicolumn{4}{|>{\centering\arraybackslash}m{12.2cm}|}{\cellcolor{\headerColor}\textbf{Curso Alterno}} \\
    \hline
      2a & \multicolumn{3}{|>{\raggedright\arraybackslash}X|}{Si el dataset es privado y el usuario no es el propietario ni administrador, se devolverá un error de permisos} \\
      \hline
\end{tabularx}
\vspace{-1em}
\begin{table}[H]
    \begin{tabularx}{\linewidth}{
      |>{\centering\arraybackslash}p{2.4cm}
      |>{\raggedright\arraybackslash}p{3cm}
      |>{\centering\arraybackslash}p{2.4cm}
      |>{\raggedright\arraybackslash}p{3cm}|
    }
        \hline
        \multicolumn{4}{|>{\centering\arraybackslash}m{12.2cm}|}{\cellcolor{\headerColor}\textbf{Otros Datos}} \\
        \hline
        \textbf{Frecuencia esperada} & Alta & \textbf{Rendimiento} & Medio \\
        \hline
        \textbf{Importancia} & Alta & \textbf{Urgencia} & Media \\
        \hline
        \textbf{Estado} & Realizado & \textbf{Estabilidad} & Alta \\
        \hline
        \multicolumn{4}{|>{\centering\arraybackslash}m{12.2cm}|}{\cellcolor{\headerColor}\textbf{Comentarios}} \\
        \hline
        \multicolumn{4}{|>{\centering\arraybackslash}X|}{}\\
        \hline
    \end{tabularx}
\end{table}
\subsubsection{Caso de Uso: Subir Modelo}
\begin{table}[H]
    \renewcommand{\arraystretch}{1.3}
    \begin{tabularx}{\linewidth}{|>{\centering\arraybackslash}m{2.5cm}|>{\centering\arraybackslash}m{6.9cm}|>{\centering\arraybackslash}m{1.94cm}|}
        \hline
        \rowcolor{\headerColor}\textbf{Caso de Uso} & \textbf{Subir Modelo} & \textbf{CU-10} \\
        \hline
        \textbf{Actores} & \multicolumn{2}{|>{\raggedright\arraybackslash}X|}{Usuario}\\
        \hline
        \textbf{Tipo} & \multicolumn{2}{|>{\raggedright\arraybackslash}X|}{Primario} \\
        \hline
   \end{tabularx}
   \vspace{-1.1em}
  \begin{tabularx}{\linewidth}{|>{\centering\arraybackslash}m{2.5cm}|>{\centering\arraybackslash}m{4.42cm}|>{\centering\arraybackslash}m{4.42cm}|}
      \textbf{Referencias} & RF-9, RNF-3 & \\
      \hline
      \textbf{Precondición} & \multicolumn{2}{|>{\raggedright\arraybackslash}X|}{El modelo debe existir en HuggingFace y el usuario no debe haber subido ya ese modelo al sistema} \\
      \hline
      \textbf{Postcondición} & \multicolumn{2}{|>{\raggedright\arraybackslash}X|}{Se añadirá el modelo al sistema} \\
      \hline
    \end{tabularx}
\end{table}
\vspace{-1em}
\begin{table}[H]
    \begin{tabularx}{\linewidth}{|>{\centering\arraybackslash}m{12.2cm}|}
      \hline
      \rowcolor{\headerColor}\textbf{Propósito} \\
      \hline
      Permitir al usuario subir un modelo de clasificación al sistema \\
      \hline
    \end{tabularx}
\end{table}
\vspace{-1em}
\begin{table}[H]
    \begin{tabularx}{\linewidth}{|>{\centering\arraybackslash}m{12.2cm}|}
      \hline
      \rowcolor{\headerColor}\textbf{Resumen} \\
      \hline
      El usuario subirá un modelo al sistema. Para ello podrá buscar en HuggingFace un modelo existente y seleccionarlo. \\
      \hline
    \end{tabularx}
\end{table}
\vspace{-1em}
\begin{tabularx}{\linewidth}{
    |>{\centering\arraybackslash}p{0.3cm}
    |>{\raggedright\arraybackslash}p{5.1cm}
    |>{\centering\arraybackslash}p{0.3cm}
    |>{\raggedright\arraybackslash}p{5.1cm}|
  }
    \hline
    \multicolumn{4}{|>{\centering\arraybackslash}m{12.2cm}|}{\cellcolor{\headerColor}\textbf{Curso Normal}} \\
    \hline
    \endfirsthead
      1 & Usuario: Seleccionará añadir un nuevo modelo &  &  \\
      \hline
      2 & Usuario: Introducirá el nombre del modelo en el campo de búsqueda &  &  \\
      \hline
       &  & 3 & El sistema buscará los modelos que tengan ese nombre en HHuggingFace \\
      \hline
      4 & Usuario: recibirá una lista con los modelos que contienen el nombre buscado &  &  \\
      \hline
      5 & Usuario: Seleccionará uno de los modelos &  &  \\
      \hline
       &  & 6 & El sistema guardará el modelo seleccionado en el sistema \\
      \hline
      7 & Usuario: Recibirá una respuesta existosa &  &  \\
      \hline
    \multicolumn{4}{|>{\centering\arraybackslash}m{12.2cm}|}{\cellcolor{\headerColor}\textbf{Curso Alterno}} \\
    \hline
      3a & \multicolumn{3}{|>{\raggedright\arraybackslash}X|}{Si no se encuentra ningún modelo con el nombre proporcionado el sistema informará de ello al usuario y volverá al paso 2} \\
      \hline
      6a & \multicolumn{3}{|>{\raggedright\arraybackslash}X|}{Si el modelo ya está subido en el sistema por el usuario se devolverá un mensaje de error al usuario} \\
      \hline
\end{tabularx}
\vspace{-1em}
\begin{table}[H]
    \begin{tabularx}{\linewidth}{
      |>{\centering\arraybackslash}p{2.4cm}
      |>{\raggedright\arraybackslash}p{3cm}
      |>{\centering\arraybackslash}p{2.4cm}
      |>{\raggedright\arraybackslash}p{3cm}|
    }
        \hline
        \multicolumn{4}{|>{\centering\arraybackslash}m{12.2cm}|}{\cellcolor{\headerColor}\textbf{Otros Datos}} \\
        \hline
        \textbf{Frecuencia esperada} & Media & \textbf{Rendimiento} & Alto \\
        \hline
        \textbf{Importancia} & Alta & \textbf{Urgencia} & Alta \\
        \hline
        \textbf{Estado} & Realizado & \textbf{Estabilidad} & Alta \\
        \hline
        \multicolumn{4}{|>{\centering\arraybackslash}m{12.2cm}|}{\cellcolor{\headerColor}\textbf{Comentarios}} \\
        \hline
        \multicolumn{4}{|>{\centering\arraybackslash}X|}{}\\
        \hline
    \end{tabularx}
\end{table}
\subsubsection{Caso de Uso: Borrar Modelo}
\begin{table}[H]
    \renewcommand{\arraystretch}{1.3}
    \begin{tabularx}{\linewidth}{|>{\centering\arraybackslash}m{2.5cm}|>{\centering\arraybackslash}m{6.9cm}|>{\centering\arraybackslash}m{1.94cm}|}
        \hline
        \rowcolor{\headerColor}\textbf{Caso de Uso} & \textbf{Borrar Modelo} & \textbf{CU-11} \\
        \hline
        \textbf{Actores} & \multicolumn{2}{|>{\raggedright\arraybackslash}X|}{Usuario}\\
        \hline
        \textbf{Tipo} & \multicolumn{2}{|>{\raggedright\arraybackslash}X|}{Primario} \\
        \hline
   \end{tabularx}
   \vspace{-1.1em}
  \begin{tabularx}{\linewidth}{|>{\centering\arraybackslash}m{2.5cm}|>{\centering\arraybackslash}m{4.42cm}|>{\centering\arraybackslash}m{4.42cm}|}
      \textbf{Referencias} & RF-16-RF-19 & CU-02\\
      \hline
      \textbf{Precondición} & \multicolumn{2}{|>{\raggedright\arraybackslash}X|}{El modelo debe existir en el sistema} \\
      \hline
      \textbf{Postcondición} & \multicolumn{2}{|>{\raggedright\arraybackslash}X|}{Se borrará el modelo del sistema} \\
      \hline
    \end{tabularx}
\end{table}
\vspace{-1em}
\begin{table}[H]
    \begin{tabularx}{\linewidth}{|>{\centering\arraybackslash}m{12.2cm}|}
      \hline
      \rowcolor{\headerColor}\textbf{Propósito} \\
      \hline
      Permitir al usuario borrar los modelos \\
      \hline
    \end{tabularx}
\end{table}
\vspace{-1em}
\begin{table}[H]
    \begin{tabularx}{\linewidth}{|>{\centering\arraybackslash}m{12.2cm}|}
      \hline
      \rowcolor{\headerColor}\textbf{Resumen} \\
      \hline
      Un usuario o administrador podrá borrar un dataset existente en el sistema \\
      \hline
    \end{tabularx}
\end{table}
\vspace{-1em}
\begin{tabularx}{\linewidth}{
    |>{\centering\arraybackslash}p{0.3cm}
    |>{\raggedright\arraybackslash}p{5.1cm}
    |>{\centering\arraybackslash}p{0.3cm}
    |>{\raggedright\arraybackslash}p{5.1cm}|
  }
    \hline
    \multicolumn{4}{|>{\centering\arraybackslash}m{12.2cm}|}{\cellcolor{\headerColor}\textbf{Curso Normal}} \\
    \hline
    \endfirsthead
      1 & Usuario: Seleccionará el modelo a borrar y se le solicitará confirmación antes de borrarlo &  &  \\
      \hline
       &  & 2 & Se borrará el modelo de la base de datos \\
      \hline
      3 & Usuario: Recibirá una respuesta exitosa &  &  \\
      \hline
    \multicolumn{4}{|>{\centering\arraybackslash}m{12.2cm}|}{\cellcolor{\headerColor}\textbf{Curso Alterno}} \\
    \hline
       & \multicolumn{3}{|>{\raggedright\arraybackslash}X|}{} \\
      \hline
\end{tabularx}
\vspace{-1em}
\begin{table}[H]
    \begin{tabularx}{\linewidth}{
      |>{\centering\arraybackslash}p{2.4cm}
      |>{\raggedright\arraybackslash}p{3cm}
      |>{\centering\arraybackslash}p{2.4cm}
      |>{\raggedright\arraybackslash}p{3cm}|
    }
        \hline
        \multicolumn{4}{|>{\centering\arraybackslash}m{12.2cm}|}{\cellcolor{\headerColor}\textbf{Otros Datos}} \\
        \hline
        \textbf{Frecuencia esperada} & Baja & \textbf{Rendimiento} & Alta \\
        \hline
        \textbf{Importancia} & Alta & \textbf{Urgencia} & Baja \\
        \hline
        \textbf{Estado} & Realizado & \textbf{Estabilidad} & Alta \\
        \hline
        \multicolumn{4}{|>{\centering\arraybackslash}m{12.2cm}|}{\cellcolor{\headerColor}\textbf{Comentarios}} \\
        \hline
        \multicolumn{4}{|>{\centering\arraybackslash}X|}{}\\
        \hline
    \end{tabularx}
\end{table}
\subsubsection{Caso de Uso: Listar Modelos}
\begin{table}[H]
    \renewcommand{\arraystretch}{1.3}
    \begin{tabularx}{\linewidth}{|>{\centering\arraybackslash}m{2.5cm}|>{\centering\arraybackslash}m{6.9cm}|>{\centering\arraybackslash}m{1.94cm}|}
        \hline
        \rowcolor{\headerColor}\textbf{Caso de Uso} & \textbf{Listar Modelos} & \textbf{CU-12} \\
        \hline
        \textbf{Actores} & \multicolumn{2}{|>{\raggedright\arraybackslash}X|}{Usuario}\\
        \hline
        \textbf{Tipo} & \multicolumn{2}{|>{\raggedright\arraybackslash}X|}{Primario} \\
        \hline
   \end{tabularx}
   \vspace{-1.1em}
  \begin{tabularx}{\linewidth}{|>{\centering\arraybackslash}m{2.5cm}|>{\centering\arraybackslash}m{4.42cm}|>{\centering\arraybackslash}m{4.42cm}|}
      \textbf{Referencias} & RF-13 & \\
      \hline
      \textbf{Precondición} & \multicolumn{2}{|>{\raggedright\arraybackslash}X|}{El usuario del que se quieren listar los datasets debe existir} \\
      \hline
      \textbf{Postcondición} & \multicolumn{2}{|>{\raggedright\arraybackslash}X|}{} \\
      \hline
    \end{tabularx}
\end{table}
\vspace{-1em}
\begin{table}[H]
    \begin{tabularx}{\linewidth}{|>{\centering\arraybackslash}m{12.2cm}|}
      \hline
      \rowcolor{\headerColor}\textbf{Propósito} \\
      \hline
      Mostrar el listado de modelos que ha subido un usuario. \\
      \hline
    \end{tabularx}
\end{table}
\vspace{-1em}
\begin{table}[H]
    \begin{tabularx}{\linewidth}{|>{\centering\arraybackslash}m{12.2cm}|}
      \hline
      \rowcolor{\headerColor}\textbf{Resumen} \\
      \hline
      Un usuario podrá ver el listado de modelos de cualquier usuario. \\
      \hline
    \end{tabularx}
\end{table}
\vspace{-1em}
\begin{tabularx}{\linewidth}{
    |>{\centering\arraybackslash}p{0.3cm}
    |>{\raggedright\arraybackslash}p{5.1cm}
    |>{\centering\arraybackslash}p{0.3cm}
    |>{\raggedright\arraybackslash}p{5.1cm}|
  }
    \hline
    \multicolumn{4}{|>{\centering\arraybackslash}m{12.2cm}|}{\cellcolor{\headerColor}\textbf{Curso Normal}} \\
    \hline
    \endfirsthead
      1 & Usuario: Seleccionará ver los modelos de un usuario en concreto &  &  \\
      \hline
       &  & 2 & Se obtendrán de la base de datos la lista de modelos subidos por el usuario objetivo \\
      \hline
      3 & Usuario: Recibirá una lista de modelos que ha subido el usuario especificado &  &  \\
      \hline
    \multicolumn{4}{|>{\centering\arraybackslash}m{12.2cm}|}{\cellcolor{\headerColor}\textbf{Curso Alterno}} \\
    \hline
       & \multicolumn{3}{|>{\raggedright\arraybackslash}X|}{} \\
      \hline
\end{tabularx}
\vspace{-1em}
\begin{table}[H]
    \begin{tabularx}{\linewidth}{
      |>{\centering\arraybackslash}p{2.4cm}
      |>{\raggedright\arraybackslash}p{3cm}
      |>{\centering\arraybackslash}p{2.4cm}
      |>{\raggedright\arraybackslash}p{3cm}|
    }
        \hline
        \multicolumn{4}{|>{\centering\arraybackslash}m{12.2cm}|}{\cellcolor{\headerColor}\textbf{Otros Datos}} \\
        \hline
        \textbf{Frecuencia esperada} & Alta & \textbf{Rendimiento} & Alto \\
        \hline
        \textbf{Importancia} & Alta & \textbf{Urgencia} & Alta \\
        \hline
        \textbf{Estado} & Realizado & \textbf{Estabilidad} & Alta \\
        \hline
        \multicolumn{4}{|>{\centering\arraybackslash}m{12.2cm}|}{\cellcolor{\headerColor}\textbf{Comentarios}} \\
        \hline
        \multicolumn{4}{|>{\centering\arraybackslash}X|}{}\\
        \hline
    \end{tabularx}
\end{table}
\subsubsection{Caso de Uso: Realizar Clasificación}
\begin{table}[H]
    \renewcommand{\arraystretch}{1.3}
    \begin{tabularx}{\linewidth}{|>{\centering\arraybackslash}m{2.5cm}|>{\centering\arraybackslash}m{6.9cm}|>{\centering\arraybackslash}m{1.94cm}|}
        \hline
        \rowcolor{\headerColor}\textbf{Caso de Uso} & \textbf{Realizar Clasificación} & \textbf{CU-13} \\
        \hline
        \textbf{Actores} & \multicolumn{2}{|>{\raggedright\arraybackslash}X|}{Usuario Propietario, Administrador}\\
        \hline
        \textbf{Tipo} & \multicolumn{2}{|>{\raggedright\arraybackslash}X|}{Primario} \\
        \hline
   \end{tabularx}
   \vspace{-1.1em}
  \begin{tabularx}{\linewidth}{|>{\centering\arraybackslash}m{2.5cm}|>{\centering\arraybackslash}m{4.42cm}|>{\centering\arraybackslash}m{4.42cm}|}
      \textbf{Referencias} & RF-10, RF-13, RNF-4 & \\
      \hline
      \textbf{Precondición} & \multicolumn{2}{|>{\raggedright\arraybackslash}X|}{El dataset al que se quiere realizar la clasificación debe existir en el sistema} \\
      \hline
      \textbf{Postcondición} & \multicolumn{2}{|>{\raggedright\arraybackslash}X|}{Se creará una tarea para ejecutar la clasificación} \\
      \hline
    \end{tabularx}
\end{table}
\vspace{-1em}
\begin{table}[H]
    \begin{tabularx}{\linewidth}{|>{\centering\arraybackslash}m{12.2cm}|}
      \hline
      \rowcolor{\headerColor}\textbf{Propósito} \\
      \hline
      Clasificar un dataset \\
      \hline
    \end{tabularx}
\end{table}
\vspace{-1em}
\begin{table}[H]
    \begin{tabularx}{\linewidth}{|>{\centering\arraybackslash}m{12.2cm}|}
      \hline
      \rowcolor{\headerColor}\textbf{Resumen} \\
      \hline
      Un usuario podrá ejecutar una clasificación de la que sea propietario \\
      \hline
    \end{tabularx}
\end{table}
\vspace{-1em}
\begin{tabularx}{\linewidth}{
    |>{\centering\arraybackslash}p{0.3cm}
    |>{\raggedright\arraybackslash}p{5.1cm}
    |>{\centering\arraybackslash}p{0.3cm}
    |>{\raggedright\arraybackslash}p{5.1cm}|
  }
    \hline
    \multicolumn{4}{|>{\centering\arraybackslash}m{12.2cm}|}{\cellcolor{\headerColor}\textbf{Curso Normal}} \\
    \hline
    \endfirsthead
      1 & Usuario: Seleccionará realizar una nueva clasificación de un dataset. &  &  \\
      \hline
       &  & 2 & El sistema obtendrá todos los modelos subidos por el usuario. \\
      \hline
       &  & 3 & El sistema obtendrá todos los nombres de las columnas del dataset. \\
      \hline
      4 & Usuario: Irá seleccionando los parámetros de la clasificación de uno en uno hasta que seleccione la columna en la que se encuentran las clases del dataset. &  &  \\
      \hline
       &  & 5 & El sistema obtendrá los valores diferentes de esa columna y le permitirá al usuario escribir una descripción por cada clase. \\
      \hline
      6 & Usuario: Seleccionará el resto de parámetros de la clasificación. &  &  \\
      \hline
       &  & 7 & El sistema guardará la clasificación en la base de datos y pondrá la tarea en cola para que el gestor de tareas en background la procese cuando esté disponible \\
      \hline
      8 & Usuario: Recibirá una respuesta exitosa del sistema. &  &  \\
      \hline
       &  & 9 & El gestor de colas obtendrá la tarea de la cola y actualizará el estado de la clasificación. \\
      \hline
       &  & 10 & Una vez terminada la clasificación, se guardará el resultado en disco y se actualizará el estado en la base de datos. \\
      \hline
    \multicolumn{4}{|>{\centering\arraybackslash}m{12.2cm}|}{\cellcolor{\headerColor}\textbf{Curso Alterno}} \\
    \hline
      5a & \multicolumn{3}{|>{\raggedright\arraybackslash}X|}{Si la columna seleccionada tiene demasiados valores diferentes se pedirá al usuario que seleccione otra.} \\
      \hline
\end{tabularx}
\vspace{-1em}
\begin{table}[H]
    \begin{tabularx}{\linewidth}{
      |>{\centering\arraybackslash}p{2.4cm}
      |>{\raggedright\arraybackslash}p{3cm}
      |>{\centering\arraybackslash}p{2.4cm}
      |>{\raggedright\arraybackslash}p{3cm}|
    }
        \hline
        \multicolumn{4}{|>{\centering\arraybackslash}m{12.2cm}|}{\cellcolor{\headerColor}\textbf{Otros Datos}} \\
        \hline
        \textbf{Frecuencia esperada} & Media & \textbf{Rendimiento} & Bajo \\
        \hline
        \textbf{Importancia} & Alta & \textbf{Urgencia} & Alta \\
        \hline
        \textbf{Estado} & Realizado & \textbf{Estabilidad} & Alta \\
        \hline
        \multicolumn{4}{|>{\centering\arraybackslash}m{12.2cm}|}{\cellcolor{\headerColor}\textbf{Comentarios}} \\
        \hline
        \multicolumn{4}{|>{\centering\arraybackslash}X|}{La creación de la clasificación, la puesta en cola y la respuesta al usuario deben ser rápidas, pero el tiempo que tarde en realizarse la clasificación dependerá del número de datos con los que se quiera realizar la clasificación.}\\
        \hline
    \end{tabularx}
\end{table}
\subsubsection{Caso de Uso: Borrar Clasificación}
\begin{table}[H]
    \renewcommand{\arraystretch}{1.3}
    \begin{tabularx}{\linewidth}{|>{\centering\arraybackslash}m{2.5cm}|>{\centering\arraybackslash}m{6.9cm}|>{\centering\arraybackslash}m{1.94cm}|}
        \hline
        \rowcolor{\headerColor}\textbf{Caso de Uso} & \textbf{Borrar Clasificación} & \textbf{CU-14} \\
        \hline
        \textbf{Actores} & \multicolumn{2}{|>{\raggedright\arraybackslash}X|}{Usuario Propietario, Administrador}\\
        \hline
        \textbf{Tipo} & \multicolumn{2}{|>{\raggedright\arraybackslash}X|}{Primario} \\
        \hline
   \end{tabularx}
   \vspace{-1.1em}
  \begin{tabularx}{\linewidth}{|>{\centering\arraybackslash}m{2.5cm}|>{\centering\arraybackslash}m{4.42cm}|>{\centering\arraybackslash}m{4.42cm}|}
      \textbf{Referencias} & RF-17, RF-19 & CU-02\\
      \hline
      \textbf{Precondición} & \multicolumn{2}{|>{\raggedright\arraybackslash}X|}{La clasificación tiene que existir en el sistema} \\
      \hline
      \textbf{Postcondición} & \multicolumn{2}{|>{\raggedright\arraybackslash}X|}{La clasificación se borrará del sistema} \\
      \hline
    \end{tabularx}
\end{table}
\vspace{-1em}
\begin{table}[H]
    \begin{tabularx}{\linewidth}{|>{\centering\arraybackslash}m{12.2cm}|}
      \hline
      \rowcolor{\headerColor}\textbf{Propósito} \\
      \hline
      Borrar una clasificación de un usuario \\
      \hline
    \end{tabularx}
\end{table}
\vspace{-1em}
\begin{table}[H]
    \begin{tabularx}{\linewidth}{|>{\centering\arraybackslash}m{12.2cm}|}
      \hline
      \rowcolor{\headerColor}\textbf{Resumen} \\
      \hline
      Un usuario o administrador podrá borrar una clasificación existente en el sistema \\
      \hline
    \end{tabularx}
\end{table}
\vspace{-1em}
\begin{tabularx}{\linewidth}{
    |>{\centering\arraybackslash}p{0.3cm}
    |>{\raggedright\arraybackslash}p{5.1cm}
    |>{\centering\arraybackslash}p{0.3cm}
    |>{\raggedright\arraybackslash}p{5.1cm}|
  }
    \hline
    \multicolumn{4}{|>{\centering\arraybackslash}m{12.2cm}|}{\cellcolor{\headerColor}\textbf{Curso Normal}} \\
    \hline
    \endfirsthead
      1 & Usuario: Seleccionará la clasificación a borrar y se le solicitará confirmación antes de borrarlo &  &  \\
      \hline
       &  & 2 & Se borrarán los ficheros físicos asociados a la clasificación \\
      \hline
       &  & 3 & Se borrará la clasificación de la base de datos \\
      \hline
      4 & Usuario: Recibirá una respuesta exitosa &  &  \\
      \hline
    \multicolumn{4}{|>{\centering\arraybackslash}m{12.2cm}|}{\cellcolor{\headerColor}\textbf{Curso Alterno}} \\
    \hline
       & \multicolumn{3}{|>{\raggedright\arraybackslash}X|}{} \\
      \hline
\end{tabularx}
\vspace{-1em}
\begin{table}[H]
    \begin{tabularx}{\linewidth}{
      |>{\centering\arraybackslash}p{2.4cm}
      |>{\raggedright\arraybackslash}p{3cm}
      |>{\centering\arraybackslash}p{2.4cm}
      |>{\raggedright\arraybackslash}p{3cm}|
    }
        \hline
        \multicolumn{4}{|>{\centering\arraybackslash}m{12.2cm}|}{\cellcolor{\headerColor}\textbf{Otros Datos}} \\
        \hline
        \textbf{Frecuencia esperada} & Baja & \textbf{Rendimiento} & Alto \\
        \hline
        \textbf{Importancia} & Alta & \textbf{Urgencia} & Baja \\
        \hline
        \textbf{Estado} & Realizado & \textbf{Estabilidad} & Alta \\
        \hline
        \multicolumn{4}{|>{\centering\arraybackslash}m{12.2cm}|}{\cellcolor{\headerColor}\textbf{Comentarios}} \\
        \hline
        \multicolumn{4}{|>{\centering\arraybackslash}X|}{}\\
        \hline
    \end{tabularx}
\end{table}
\subsubsection{Caso de Uso: Listar Clasificaciones de un usuario}
\begin{table}[H]
    \renewcommand{\arraystretch}{1.3}
    \begin{tabularx}{\linewidth}{|>{\centering\arraybackslash}m{2.5cm}|>{\centering\arraybackslash}m{6.9cm}|>{\centering\arraybackslash}m{1.94cm}|}
        \hline
        \rowcolor{\headerColor}\textbf{Caso de Uso} & \textbf{Listar Clasificaciones de un usuario} & \textbf{CU-15} \\
        \hline
        \textbf{Actores} & \multicolumn{2}{|>{\raggedright\arraybackslash}X|}{Usuario}\\
        \hline
        \textbf{Tipo} & \multicolumn{2}{|>{\raggedright\arraybackslash}X|}{Primario} \\
        \hline
   \end{tabularx}
   \vspace{-1.1em}
  \begin{tabularx}{\linewidth}{|>{\centering\arraybackslash}m{2.5cm}|>{\centering\arraybackslash}m{4.42cm}|>{\centering\arraybackslash}m{4.42cm}|}
      \textbf{Referencias} & RF-14 & \\
      \hline
      \textbf{Precondición} & \multicolumn{2}{|>{\raggedright\arraybackslash}X|}{El usuario del que se quieren listar las clasificaciones debe existir} \\
      \hline
      \textbf{Postcondición} & \multicolumn{2}{|>{\raggedright\arraybackslash}X|}{} \\
      \hline
    \end{tabularx}
\end{table}
\vspace{-1em}
\begin{table}[H]
    \begin{tabularx}{\linewidth}{|>{\centering\arraybackslash}m{12.2cm}|}
      \hline
      \rowcolor{\headerColor}\textbf{Propósito} \\
      \hline
      Mostrar el listado de clasificaciones que ha subido un usuario \\
      \hline
    \end{tabularx}
\end{table}
\vspace{-1em}
\begin{table}[H]
    \begin{tabularx}{\linewidth}{|>{\centering\arraybackslash}m{12.2cm}|}
      \hline
      \rowcolor{\headerColor}\textbf{Resumen} \\
      \hline
      Un usuario podrá ver el listado de clasificaciones de cualquier usuario, siempre que sean clasificaciones de datasets públicos o sea propietario \\
      \hline
    \end{tabularx}
\end{table}
\vspace{-1em}
\begin{tabularx}{\linewidth}{
    |>{\centering\arraybackslash}p{0.3cm}
    |>{\raggedright\arraybackslash}p{5.1cm}
    |>{\centering\arraybackslash}p{0.3cm}
    |>{\raggedright\arraybackslash}p{5.1cm}|
  }
    \hline
    \multicolumn{4}{|>{\centering\arraybackslash}m{12.2cm}|}{\cellcolor{\headerColor}\textbf{Curso Normal}} \\
    \hline
    \endfirsthead
      1 & Usuario: Seleccionará ver las clasificaciones de un usuario en concreto &  &  \\
      \hline
       &  & 2 & Se comprobará que el usuario que ha solicitado las clasificaciones es administrador o es propietario de las clasificaciones solicitadas \\
      \hline
       &  & 3 & Se obtendrán de la base de datos la lista de clasificaciones relacionados con el usuario objetivo a los que el usuario solicitante tiene permiso \\
      \hline
      4 & Usuario: Recibirá una lista de clasificaciones que ha subido el usuario especificado &  &  \\
      \hline
    \multicolumn{4}{|>{\centering\arraybackslash}m{12.2cm}|}{\cellcolor{\headerColor}\textbf{Curso Alterno}} \\
    \hline
       & \multicolumn{3}{|>{\raggedright\arraybackslash}X|}{} \\
      \hline
\end{tabularx}
\vspace{-1em}
\begin{table}[H]
    \begin{tabularx}{\linewidth}{
      |>{\centering\arraybackslash}p{2.4cm}
      |>{\raggedright\arraybackslash}p{3cm}
      |>{\centering\arraybackslash}p{2.4cm}
      |>{\raggedright\arraybackslash}p{3cm}|
    }
        \hline
        \multicolumn{4}{|>{\centering\arraybackslash}m{12.2cm}|}{\cellcolor{\headerColor}\textbf{Otros Datos}} \\
        \hline
        \textbf{Frecuencia esperada} & Alta & \textbf{Rendimiento} & Alto \\
        \hline
        \textbf{Importancia} & Alta & \textbf{Urgencia} & Media \\
        \hline
        \textbf{Estado} & Realizado & \textbf{Estabilidad} & Alta \\
        \hline
        \multicolumn{4}{|>{\centering\arraybackslash}m{12.2cm}|}{\cellcolor{\headerColor}\textbf{Comentarios}} \\
        \hline
        \multicolumn{4}{|>{\centering\arraybackslash}X|}{}\\
        \hline
    \end{tabularx}
\end{table}
\subsubsection{Caso de Uso: Listar Clasificaciones de un dataset}
\begin{table}[H]
    \renewcommand{\arraystretch}{1.3}
    \begin{tabularx}{\linewidth}{|>{\centering\arraybackslash}m{2.5cm}|>{\centering\arraybackslash}m{6.9cm}|>{\centering\arraybackslash}m{1.94cm}|}
        \hline
        \rowcolor{\headerColor}\textbf{Caso de Uso} & \textbf{Listar Clasificaciones de un dataset} & \textbf{CU-16} \\
        \hline
        \textbf{Actores} & \multicolumn{2}{|>{\raggedright\arraybackslash}X|}{Usuario}\\
        \hline
        \textbf{Tipo} & \multicolumn{2}{|>{\raggedright\arraybackslash}X|}{Primario} \\
        \hline
   \end{tabularx}
   \vspace{-1.1em}
  \begin{tabularx}{\linewidth}{|>{\centering\arraybackslash}m{2.5cm}|>{\centering\arraybackslash}m{4.42cm}|>{\centering\arraybackslash}m{4.42cm}|}
      \textbf{Referencias} & RF-14 & CU-09\\
      \hline
      \textbf{Precondición} & \multicolumn{2}{|>{\raggedright\arraybackslash}X|}{El dataset del que se quieren listar las clasificaciones debe existir} \\
      \hline
      \textbf{Postcondición} & \multicolumn{2}{|>{\raggedright\arraybackslash}X|}{} \\
      \hline
    \end{tabularx}
\end{table}
\vspace{-1em}
\begin{table}[H]
    \begin{tabularx}{\linewidth}{|>{\centering\arraybackslash}m{12.2cm}|}
      \hline
      \rowcolor{\headerColor}\textbf{Propósito} \\
      \hline
      Mostrar el listado de clasificaciones realizadas a un dataset \\
      \hline
    \end{tabularx}
\end{table}
\vspace{-1em}
\begin{table}[H]
    \begin{tabularx}{\linewidth}{|>{\centering\arraybackslash}m{12.2cm}|}
      \hline
      \rowcolor{\headerColor}\textbf{Resumen} \\
      \hline
      Un usuario podrá ver el listado de clasificaciones de cualquier dataset \\
      \hline
    \end{tabularx}
\end{table}
\vspace{-1em}
\begin{tabularx}{\linewidth}{
    |>{\centering\arraybackslash}p{0.3cm}
    |>{\raggedright\arraybackslash}p{5.1cm}
    |>{\centering\arraybackslash}p{0.3cm}
    |>{\raggedright\arraybackslash}p{5.1cm}|
  }
    \hline
    \multicolumn{4}{|>{\centering\arraybackslash}m{12.2cm}|}{\cellcolor{\headerColor}\textbf{Curso Normal}} \\
    \hline
    \endfirsthead
      1 & Usuario: Seleccionará ver las clasificaciones de un dataset en concreto &  &  \\
      \hline
       &  & 2 & Se comprobará que el usuario que ha solicitado las clasificaciones es administrador o es propietario de las clasificaciones solicitadas \\
      \hline
       &  & 3 & Se obtendrán de la base de datos la lista de clasificaciones relacionados con el dataset objetivo \\
      \hline
      4 & Usuario: Recibirá una lista de clasificaciones realizadas al dataset &  &  \\
      \hline
    \multicolumn{4}{|>{\centering\arraybackslash}m{12.2cm}|}{\cellcolor{\headerColor}\textbf{Curso Alterno}} \\
    \hline
      2a & \multicolumn{3}{|>{\raggedright\arraybackslash}X|}{Si el usuario no tiene permiso para ver el dataset se devolverá un mensaje de error al usuario.} \\
      \hline
\end{tabularx}
\vspace{-1em}
\begin{table}[H]
    \begin{tabularx}{\linewidth}{
      |>{\centering\arraybackslash}p{2.4cm}
      |>{\raggedright\arraybackslash}p{3cm}
      |>{\centering\arraybackslash}p{2.4cm}
      |>{\raggedright\arraybackslash}p{3cm}|
    }
        \hline
        \multicolumn{4}{|>{\centering\arraybackslash}m{12.2cm}|}{\cellcolor{\headerColor}\textbf{Otros Datos}} \\
        \hline
        \textbf{Frecuencia esperada} & Alta & \textbf{Rendimiento} & Alto \\
        \hline
        \textbf{Importancia} & Alta & \textbf{Urgencia} & Media \\
        \hline
        \textbf{Estado} & Realizado & \textbf{Estabilidad} & Alta \\
        \hline
        \multicolumn{4}{|>{\centering\arraybackslash}m{12.2cm}|}{\cellcolor{\headerColor}\textbf{Comentarios}} \\
        \hline
        \multicolumn{4}{|>{\centering\arraybackslash}X|}{}\\
        \hline
    \end{tabularx}
\end{table}
\subsubsection{Caso de Uso: Ver Resultados de una clasificación}
\begin{table}[H]
    \renewcommand{\arraystretch}{1.3}
    \begin{tabularx}{\linewidth}{|>{\centering\arraybackslash}m{2.5cm}|>{\centering\arraybackslash}m{6.9cm}|>{\centering\arraybackslash}m{1.94cm}|}
        \hline
        \rowcolor{\headerColor}\textbf{Caso de Uso} & \textbf{Ver Resultados de una clasificación} & \textbf{CU-17} \\
        \hline
        \textbf{Actores} & \multicolumn{2}{|>{\raggedright\arraybackslash}X|}{Usuario}\\
        \hline
        \textbf{Tipo} & \multicolumn{2}{|>{\raggedright\arraybackslash}X|}{Primario} \\
        \hline
   \end{tabularx}
   \vspace{-1.1em}
  \begin{tabularx}{\linewidth}{|>{\centering\arraybackslash}m{2.5cm}|>{\centering\arraybackslash}m{4.42cm}|>{\centering\arraybackslash}m{4.42cm}|}
      \textbf{Referencias} & RF-14 & \\
      \hline
      \textbf{Precondición} & \multicolumn{2}{|>{\raggedright\arraybackslash}X|}{La clasificación tiene que existir en el sistema} \\
      \hline
      \textbf{Postcondición} & \multicolumn{2}{|>{\raggedright\arraybackslash}X|}{} \\
      \hline
    \end{tabularx}
\end{table}
\vspace{-1em}
\begin{table}[H]
    \begin{tabularx}{\linewidth}{|>{\centering\arraybackslash}m{12.2cm}|}
      \hline
      \rowcolor{\headerColor}\textbf{Propósito} \\
      \hline
      Ver el estado, los parametros y los resultados de una clasificación. \\
      \hline
    \end{tabularx}
\end{table}
\vspace{-1em}
\begin{table}[H]
    \begin{tabularx}{\linewidth}{|>{\centering\arraybackslash}m{12.2cm}|}
      \hline
      \rowcolor{\headerColor}\textbf{Resumen} \\
      \hline
      Un usuario accederá a los resultados de una clasificación, donde podrá observar los parámetros con los que se ha realizado, el estado actual y los resultados obtenidos. \\
      \hline
    \end{tabularx}
\end{table}
\vspace{-1em}
\begin{tabularx}{\linewidth}{
    |>{\centering\arraybackslash}p{0.3cm}
    |>{\raggedright\arraybackslash}p{5.1cm}
    |>{\centering\arraybackslash}p{0.3cm}
    |>{\raggedright\arraybackslash}p{5.1cm}|
  }
    \hline
    \multicolumn{4}{|>{\centering\arraybackslash}m{12.2cm}|}{\cellcolor{\headerColor}\textbf{Curso Normal}} \\
    \hline
    \endfirsthead
      1 & Usuario: Accederá a la visualización de una clasificación &  &  \\
      \hline
       &  & 2 & Se comprobará si el usuario tiene permisos para acceder a la clasificación \\
      \hline
       &  & 3 & Se obtendrá la información relacionada a la clasificación de la base de datos \\
      \hline
       &  & 4 & Se obtendrá de disco el resultado si la clasificación ha terminado \\
      \hline
      5 & Usuario: Recibirá la información relacionada con la clasificación &  &  \\
      \hline
    \multicolumn{4}{|>{\centering\arraybackslash}m{12.2cm}|}{\cellcolor{\headerColor}\textbf{Curso Alterno}} \\
    \hline
      2a & \multicolumn{3}{|>{\raggedright\arraybackslash}X|}{Si el dataset asociado es privado y el usuario no es el propietario ni administrador, se devolverá un error de permisos} \\
      \hline
\end{tabularx}
\vspace{-1em}
\begin{table}[H]
    \begin{tabularx}{\linewidth}{
      |>{\centering\arraybackslash}p{2.4cm}
      |>{\raggedright\arraybackslash}p{3cm}
      |>{\centering\arraybackslash}p{2.4cm}
      |>{\raggedright\arraybackslash}p{3cm}|
    }
        \hline
        \multicolumn{4}{|>{\centering\arraybackslash}m{12.2cm}|}{\cellcolor{\headerColor}\textbf{Otros Datos}} \\
        \hline
        \textbf{Frecuencia esperada} & Alta & \textbf{Rendimiento} & Medio \\
        \hline
        \textbf{Importancia} & Alta & \textbf{Urgencia} & Media \\
        \hline
        \textbf{Estado} & Realizado & \textbf{Estabilidad} & Alta \\
        \hline
        \multicolumn{4}{|>{\centering\arraybackslash}m{12.2cm}|}{\cellcolor{\headerColor}\textbf{Comentarios}} \\
        \hline
        \multicolumn{4}{|>{\centering\arraybackslash}X|}{}\\
        \hline
    \end{tabularx}
\end{table}
